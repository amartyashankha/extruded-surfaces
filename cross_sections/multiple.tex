\subsection{Multiple Cross Sections}
\label{sec:intervals}

\begin{definition}
\label{def:compatible}
Given two cross section intervals $\mathcal C$ and $\mathcal D$, such that $\mathcal C_F$ and $\mathcal D_I$ are equivalent,
we say that $\mathcal D$ is next-compatible with $\mathcal C$ and $\mathcal C$ is previous-compatible with $\mathcal D$.
Two cross sections $C = \langle s_1, s_2,\cdots s_n \rangle$ and $D = \langle r_1, r_2,\cdots r_m \rangle$ are equivalent
if and only if $C$ and $D$ correspond to the same sequence of segments after the deletion of all zero length segments.
\end{definition}

\begin{definition}
\label{def:cross_section_sequence}
A cross section sequence is a sequence is an ordered list of cross section intervals
$ \langle \mathcal C_1, \mathcal C_2,\cdots \mathcal C_n \rangle$,
such that $C_{i}$ is next-compatible with $C_{i-1}$ for all $i\in [n-1]$.
This is equivalent to stating that $C_{i}$ is previous-compatible with $C_{i+1}$ for all $i\in [n-1]$.
Note that we do not care about the velocities of the segments.
\end{definition}

We will represent our full construction as a valid \emph{cross section sequence}.
Given a cross section sequence $\langle \mathcal C_1, \mathcal C_2,\cdots \mathcal C_n \rangle$,
the transition from $\mathcal C_i$ to $\mathcal C_{i+1}$ corresponds to the deletion of one or more
length zero segments from $\mathcal C_i$, and the addition of one or more zero length segments to obtain $\mathcal C_{i+1}$.
Figure~\ref{fig:strip_narrowing} shows a simple example.
\graphicspath{{./figures/strip_narrowing/}}
\begin{figure}[!htb]
    \begingroup
    \def\svgwidth{0.23\textwidth}
        \captionsetup[subfigure]{width=0.22\textwidth}
        \subfloat[The trivial cross section.] {
        \input{figures/strip_narrowing/strip0.pdf_tex}
        }%
    \endgroup%
    \begingroup
    \def\svgwidth{0.23\textwidth}
        \captionsetup[subfigure]{width=0.48\textwidth}
        \subfloat[New zero length segment inserted at left endpoint. Existing segment reverses direction.] {
        \input{figures/strip_narrowing/strip10.pdf_tex}
        %}%
    %\endgroup%
    %\begingroup
    \def\svgwidth{0.23\textwidth}
        %\captionsetup[subfigure]{width=0.22\textwidth}
        %\subfloat[] {
        \input{figures/strip_narrowing/strip11.pdf_tex}
        }%
    \endgroup%
    \begingroup
    \def\svgwidth{0.23\textwidth}
        \captionsetup[subfigure]{width=0.22\textwidth}
        \subfloat[Evolution continues] {
        \input{figures/strip_narrowing/strip12.pdf_tex}
        }%
    \endgroup%
    \begingroup
    \def\svgwidth{0.23\textwidth}
        \captionsetup[subfigure]{width=0.67\textwidth}
        \subfloat[New zero length segment (red) inserted between two existing segments
                  with the same velocity as the rightmost segment.
                  Existing segments retain their direction.] {
        \input{figures/strip_narrowing/strip20.pdf_tex}
        %}%
    %\endgroup%
    %\begingroup
    \def\svgwidth{0.23\textwidth}
        %\captionsetup[subfigure]{width=0.22\textwidth}
        %\subfloat[] {
        \input{figures/strip_narrowing/strip21.pdf_tex}
        %}%
    %\endgroup%
    %\begingroup
    \def\svgwidth{0.23\textwidth}
        %\captionsetup[subfigure]{width=0.22\textwidth}
        %\subfloat[] {
        \input{figures/strip_narrowing/strip22.pdf_tex}
        }%
    \endgroup%
    %\begingroup
    %\def\svgwidth{0.23\textwidth}
        %\captionsetup[subfigure]{width=0.22\textwidth}
        %\subfloat[Evolution continues] {
        %\input{figures/strip_narrowing/strip23.pdf_tex}
        %}%
    %\endgroup%
    \begingroup
    \def\svgwidth{0.23\textwidth}
        \captionsetup[subfigure]{width=0.28\textwidth}
        \subfloat[Length of leftmost (blue) segment becomes zero.] {
        \input{figures/strip_narrowing/strip24.pdf_tex}
        }%
    \endgroup%
    \begingroup
    \def\svgwidth{0.23\textwidth}
        \captionsetup[subfigure]{width=0.48\textwidth}
        \subfloat[Leftmost (zero length) segment is deleted.
                  The remaining two segments continue in the same direction.
                  Overall strip width has been reduced] {
        \input{figures/strip_narrowing/strip30.pdf_tex}
        %}%
    %\endgroup%
    %\begingroup
    \def\svgwidth{0.23\textwidth}
        %\captionsetup[subfigure]{width=0.22\textwidth}
        %\subfloat[] {
        \input{figures/strip_narrowing/strip31.pdf_tex}
        }%
    \endgroup%
    \begingroup
    \def\svgwidth{0.48\textwidth}
        \captionsetup[subfigure]{width=0.44\textwidth}
        \subfloat[Side view of strip narrowing gadget, with layers separated. The red lines denote the boundaries of the cross section.] {
        \input{figures/strip_narrowing/strip_layers.pdf_tex}
        }%
    \endgroup%
        \caption{Cross section evolution of a strip narrowing gadget. \todo[inline]{cite} }
    \label{fig:strip_narrowing}
\end{figure}


\begin{definition}
\label{def:sequence_folding}
Given a cross section sequence $\langle \mathcal C_1, \mathcal C_2,\cdots \mathcal C_n \rangle$,
We obtain $\mathcal F_i^T$ as the folding of cross section $\mathcal C_i$.
For each $i\le n-1$, we attach $\mathcal F_i^T$ to $\mathcal F_{i+1}^T$ by gluing
the final cross section of $\mathcal C_i$ to the initial cross section of $\mathcal C_{i+1}$.
This results in the final folding of the cross section sequence.
\end{definition}

\begin{proposition}
\label{prop:creases}
In addition to the creases formed along the trajectory of joints (Proposition~\ref{prop:joint_crease}),
creases are also created when a segment changes velocity.
For instance, consider two adjacent cross sections $\mathcal C$ and $\mathcal D$,
with corresponding segments $s_C\in \mathcal C_F$ and $s_D\in \mathcal D_I$,
where $s_C$ and $s_D$ are identical except for their velocity direction.
Then, the segment $s_C$ (same as $s_D$) forms a crease in the folded state.
\end{proposition}

\subsubsection{Evolution Corresponds to Flat Paper}
\label{sec:flat}

In this section we will demonstrate that the folding formed by cross section evolution is realizable from a sheet of flat paper.
We note here that our construction may still result in self intersections.

We consider a cross section sequence $\langle \mathcal C_1, \mathcal C_2,\cdots \mathcal C_n \rangle$,
where each cross section interval $\mathcal C_i$ has evolution time $T_i$.
We then use Theorem~\ref{def:cross_section_sequence} to attach the sequence of $X\times T_i$ strips,
to form a complete $X\times T$ sheet of paper, where $T = \sum T_i$.

\begin{restatable}{thm}{main}
\label{thm:main}
Consider a cross section sequence $\mathcal C = \langle \mathcal C_1, \mathcal C_2,\cdots \mathcal C_m \rangle$ where each
cross section interval $\mathcal C_i$ evolves over time $T_i$ to form a folding $\mathcal F_i$
such that the following properties hold for all segments and joints in each of the cross sections involved.
\begin{itemize}
    \item[] \vspace{-1.6em}\UniformVelocity*
    \item[] \vspace{-1.6em}\OrthogonalVelocity*
    \item[] \vspace{-1.6em}\JointOrthogonalVelocity*
    \item[] \vspace{-1.6em}\LeftRightPace*
    \item[] \vspace{-1.6em}\SegmentOrientation*
    \item[] \vspace{-1.6em}\JointVelocity*
\end{itemize}
Then, the folding $\mathcal F_{\mathcal C}$ obtained by successively gluing the final boundary of $\mathcal F_i$ to the initial boundary
of $\mathcal F_{i+1}$ (for each $1\le i<m$), is isometric to a $X\times T$ strip of paper, where $T = \sum T_i$.
\end{restatable}
\begin{proof}
From Theorem~\ref{thm:interval_strip}, we know that each $\mathcal F_i$ is isometric to a strip $Z_i$ of size $X\times T_i$.
Further, the initial and final cross sections of $\mathcal C_i$ form the parallel sides of $Z_i$ (of length $X$).
The gluing of $n$ strips of size $X\times T_i$ forms a strip of size $X\times T$.
So, the final folding is isometric to a sheet of paper with size $X\times T$.
\end{proof}
