\section{Cross Sections}
\label{sec:cross_sections}

We introduce a new method of origami construction, using cross section disgrams.
Instead of beginning our construction from a 2-dimensional sheet of paper,
we consider a 1-dimensional cross section moving forwards in time.
A simple example is demonstrated in Figure~\ref{fig:strip_narrowing}.

\todo[inline]{Conservation of length}

\begin{definition}
A cross section is defined as an ordered list of line segments $\langle s_1,s_2,\cdots,s_n\rangle$,
such that for every segment $s_i$ (except the last),
the right endpoint of $s_i$ coincides with the left endpoint of $s_{i+1}$.
Each line segment is associated with a velocity vector $\vec v_i$ of unit magnitude.
Each segment is also associated with an unit orientation vector $\vec o_i$,
which is the unit vector from the left endpint to the right endpoint.
\end{definition}

\begin{lemma}
\label{lem:uniform_velocity}
All points on a single segment move with the same velocity $v_i$.
\end{lemma}

\subsection{Joints}
\label{sec:joints}

\begin{definition}
\label{def:joints}
A cross section with $n$ segments is also associated with a list of joints
$ \langle J_1,\dots, J_{n-1} \rangle$,
where $\vec J_i$ corresponds to the right endpoint of $s_i$ (same as left endpoint of $s_{i+1}$.
A particular joint $\vec J_i$ is associated with a left segment $s_i$,
a right segment $s_{i+1}$, and a velocity $\vec J_v$.
\end{definition}

\begin{definition}
\label{def:joint_plane}
A joint plane is the plane that coincides with both segments $l$ and $r$ associated with a particular joint $J$.
In case the two segments are coincident (i.e. $\vec{\hat o}_l = \pm \vec{\hat o}_r$),
we define the joint plane to be orthogonal to $\vec{\hat v}_l = \vec{\hat v}_r$.
\end{definition}

\begin{definition}
\label{def:joint_plane_velocity}
Consider a joint $J$ associated with segments $l$, $r$, and joint plane $\mathcal P$,
where $\vec{\hat v}_l$ and $\vec{\hat v}_r$ are the velocities of segments $l$ and $r$.
We define $\vec{\hat v}_l^\shortparallel$ and $\vec{\hat v}_l^\perp$ as components of
$\vec{\hat v}_l$, such that $\vec{\hat v}_l^\shortparallel$ is the projection of $\vec{\hat v}_l$
onto $\mathcal P$, and $\vec{\hat v}_l^\perp = \vec{\hat v}_l - \vec{\hat v}_l^\shortparallel$
(orthogonal to $\mathcal P$).
Similarly, we define $\vec{\hat v}_r^\shortparallel$ and $\vec{\hat v}_r^\perp$, as the components of $\vec{\hat v}_r$.
By Invariant~\ref{inv:orthogonal_velocity}, $\vec{\hat v}_l^\shortparallel$ and $\vec{\hat v}_r^\shortparallel$
are orthogonal to $\vec{\hat o}_l$ and $\vec{\hat o}_r$ respectively.
This uniquely determines the direction of all velocity components.
\end{definition}

Henceforth, we will refer to the $\vec{\hat v}^\shortparallel$ component as the \emph{joint plane velocity},
and the $\vec{\hat v}^\perp$ component as the \emph{joint orthogonal velocity,}

\begin{restatable}{inv}{JointOrthogonalVelocity}
\label{inv:joint_orthogonal_velocity}
For a joint $J$ associated with segments $l$ and $r$, $\vec{\hat v}_l^\perp = \vec{\hat v}_r^\perp$
i.e. the joint orthogonal velocities have to be equal, such that the joint plane moves with a fixed velocity along it's normal.
As a corollary, $ \left\| \vec{\hat v}_l^\shortparallel\right\| = \left\| \vec{\hat v}_r^\shortparallel\right\|$.
\end{restatable}

\subsection{Time Travel}
\label{sec:time_travel}

In the process of time travel, all the nodes on a segment $s$,
except the \emph{joint nodes} move with velocity $\vec{\hat v_s}$ (orthogonal to $\vec{\hat o_s}$).
For example, if we allow a single segment of length $X$ to evolve for time $T$, it will result in a $X\times T$ strip of paper.

The joint node velocity $\vec J_v$ may have a component along a correspoding segment $s$ (along $\vec{\hat o_s}$).
As a result, the lengths of segments may change (Figure~\ref{fig:trapezoid_angles}).
This can be visualized as movement of the corresponding joint along one of the segments.

\begin{definition}
\label{def:segment_length}
For ever segment $s$ in a cross section $C$, we associate a left pace $L_s$,
which indicates the rate at which $s$ shrinks from its left endpoint.
Similarly, we define a right pace $R_s$ grows from its right endpoint.
Note that both these quantities can be negative.
\end{definition}
After time $T$, the length of a segment changes by $T\cdot(R_s-L_s)$.
The length of a segment is not allowed to become negative.
For a segment $s_i$ with left joint $J^L$ and right joint $J^R$, we obtain the relations
\begin{align}
\vec J_v^L-\vec{\hat v_{s_i}} = L_{s_i}\cdot \vec{\hat o_{s_i}} && \vec J_v^R-\vec{\hat v_{s_i}} = R_{s_i}\cdot \vec{\hat o_{s_i}}
\end{align}

\begin{proposition}
\label{prop:valid_joint}
A joint $J$ corresponding to segments $s$ and $t$ is \emph{valid} if and only if the evolution
resulting from the velocities $\vec{\hat v_l}$, $\vec{\hat v_r}$, and $\vec J_v$ preserves distances between nodes.
\end{proposition}
\begin{restatable}{pro}{LeftRightPace}
\label{pro:left_right_pace}
The right pace of $s_{i}$ is equal to the left pace of $s_{i+1}$.
This is to preserve the overall length of the cross section, and the distance between any two nodes.
Furthermore, the left pace of the first segment, and the right pace of the last segment should be zero, i.e. $L_0 = R_n = 0$
This ensures that the total length of the cross section does not change.
\end{restatable}

The movement of a joint increases the length of one of its associated segments,
and decreases the length of the other segment by the same amount (this ensures that the total length is preserved).
This puts some constraints on the possible velocities of adjacent segments.

%\begin{figure}[htpb]
\begin{wrapfigure}[12]{r}{0.5\textwidth}
\vspace{-2.2em}
\graphicspath{{./notebooks/}}
    \centering
    \def\svgwidth{0.5\textwidth}
    \input{notebooks/joint_plane_velocity.pdf_tex}
    \caption{A joint with segments $l$ and $r$.
    The trajectory of the joint is shown in orange.
    The trajectories of $a$ and $b$ are shown in blue.
    The green arrows indicate $\vec v^\shortparallel$}
    \label{fig:joint_plane_velocity}
\end{wrapfigure}
%\end{figure}

Consider a joint $\vec J_i$ corresponding to segments $l = s_i$ and $r = s_{i+1}$, at time $t=0$.
Henceforth, we will refer to $L_r$ as $L$, and $R_l$ as $R$.
Without loss of genenerality, we assume that $L<0$.
At a later time $t$, let the new joint position be $\vec J'_i$.
We define nodes $a$ and $b$ corresponding to $\vec J_i$ and $\vec J_{i+1}$ respectively.
We also define the initial and final positions of $a$ as $\vec a$,
and $\vec a'$, and similarly for $b$, we define $\vec b$ and $\vec b'$.
Let $d$ be the \emph{separation} between nodes $a$ and $b$.
this setup is shown in Figure~\ref{fig:joint_plane_velocity}

First, note that $\vec a, \vec b$ lie on the segment $l$, and $\vec a', \vec b'$ lie on the segment $r$,
which implies that $\vec b-\vec a = d\cdot \vec{\hat o_l}$, and $\vec b'-\vec a' = d\cdot \vec{\hat o_r}$.
\begin{align}
\vec b' - \vec a' &= (\vec b + t\cdot \vec{\hat v_r}^\shortparallel) - (\vec a + t\cdot \vec{\hat v_l}^\shortparallel)\\
\implies \vec b' - \vec a' &= (\vec b - \vec a) + t\cdot(\vec{\hat v_r}^\shortparallel - \vec{\hat v_l}^\shortparallel)\\
\implies d\cdot \vec{\hat o_r} &= d\cdot \vec{\hat o_l} + t\cdot(\vec{\hat v_r}^\shortparallel - \vec{\hat v_l}^\shortparallel)\\
\implies \vec{\hat v_r}^\shortparallel - \vec{\hat v_l}^\shortparallel &= \frac{d}{t}\cdot(\vec{\hat o_r}-\vec{\hat o_l})
= -R\cdot(\vec{\hat o_r}-\vec{\hat o_l}) = -L\cdot(\vec{\hat o_r}-\vec{\hat o_l})
\end{align}
This is only possible if $\vec{\hat o_l}\times \vec{\hat o_r}$ is oriented opposite to $\vec{\hat v_l}^\shortparallel\times \vec{\hat v_r}^\shortparallel$.

\begin{restatable}{pro}{SegmentOrientation}
\label{pro:SegmentOrientation}
Given two adjacent segments $l$ and $r$ in a cross section $C$, the vector
$\vec{\hat o_l}\times \vec{\hat o_r}$ must be oriented opposite to $\vec{\hat v_l}^\shortparallel\times \vec{\hat v_r}^\shortparallel$.
\end{restatable}

If the angle between the segments (between $\vec{\hat o_l}$ and $\vec{\hat o_r}$) is $\theta$,
the magnitude of $\vec{\hat o_r}-\vec{\hat o_l}$ is $\sqrt{2-2\cos(\theta)}$,
and $ \left\| \vec{\hat v_r}^\shortparallel-\vec{\hat v_l}^\shortparallel\right\| = v\cdot\sqrt{2-2\cos(\pi-\theta)}$.
Here, $v$ is magnitude of the plane velocity of $J_i$. Given $\phi = \theta/2$, we get --
\begin{align}
-L = -R &= \frac dt = \vec{\hat v_l} - \vec{\hat o_l}\frac{ \left\| \vec{\hat v_r}^\shortparallel
- \vec{\hat v_l}^\shortparallel\right\|}{ \left\| \vec{\hat o_r}-\vec{\hat o_l}\right\|}
&= v\cdot\frac{\sqrt{\sin^2\left( {\pi/2-\theta/2}\right)}}{\sqrt{\sin^2{\theta/2}}}
= v\cdot\cot\left( \phi \right)
\end{align}

\begin{restatable}{pro}{JointVelocity}
\label{pro:joint_velocity}
The velocity of a joint $J$ associated with segments $l$ and $r$, is a constant vector
$\vec J_v
%= \vec{\hat v_l} - \vec{\hat o_l}\frac{ \left\| \vec{\hat v_r}^\shortparallel
%- \vec{\hat v_l}^\shortparallel\right\|}{ \left\| \vec{\hat o_r}-\vec{\hat o_l}\right\|}
= \vec{\hat v_l} - \left\| \vec{\hat v_l}^\shortparallel\right\|\cdot\vec{\hat o_l}\cdot\cot\left( \theta/2 \right)$,
where $\theta$ is the angle between $\vec{\hat o_l}$ and $\vec{\hat o_r}$.
\end{restatable}

%\todo[inline]{
%Note that time evolution is reversible.
%}

\subsection{Multiple Cross Sections}
\label{sec:intervals}
\graphicspath{{./figures/strip_narrowing/}}
\begin{figure}[]
    \centering
    \begingroup
    \def\svgwidth{0.21\textwidth}
        \captionsetup[subfigure]{width=0.22\textwidth}
        \subfloat[The trivial cross section.]{\centering
        \input{figures/strip_narrowing/strip0.pdf_tex}
        }%
    \endgroup
    \hfill
    \begingroup
    \def\svgwidth{0.21\textwidth}
        \captionsetup[subfigure]{width=0.48\textwidth}
        \subfloat[New zero length segment inserted at left endpoint. Existing segment reverses velocity.]{\centering
        \input{figures/strip_narrowing/strip10.pdf_tex}
        %}%
    %\endgroup%
    %\begingroup
    \def\svgwidth{0.21\textwidth}
        %\captionsetup[subfigure]{width=0.22\textwidth}
        %\subfloat[] {
        \input{figures/strip_narrowing/strip11.pdf_tex}
        }%
    \endgroup
    \hfill
    \begingroup
    \def\svgwidth{0.21\textwidth}
        \captionsetup[subfigure]{width=0.24\textwidth}
        \subfloat[Evolution continues]{%
          \makebox[0.24\textwidth]{\input{figures/strip_narrowing/strip12.pdf_tex}}
        }%
    \endgroup%

    \vspace{-1.2em}

    \begingroup
    \def\svgwidth{0.21\textwidth}
        \captionsetup[subfigure]{width=0.67\textwidth}
        \subfloat[New zero length segment (red) inserted between two existing segments
                  with the same velocity as the rightmost segment.
                  Existing segments retain their velocites.] {
        \input{figures/strip_narrowing/strip20.pdf_tex}
        %}%
    %\endgroup%
    %\begingroup
    \def\svgwidth{0.21\textwidth}
        %\captionsetup[subfigure]{width=0.22\textwidth}
        %\subfloat[] {
        \input{figures/strip_narrowing/strip21.pdf_tex}
        %}%
    %\endgroup%
    %\begingroup
    \def\svgwidth{0.21\textwidth}
        %\captionsetup[subfigure]{width=0.22\textwidth}
        %\subfloat[] {
        \input{figures/strip_narrowing/strip22.pdf_tex}
        }%
    \endgroup%
    %\begingroup
    %\def\svgwidth{0.21\textwidth}
        %\captionsetup[subfigure]{width=0.22\textwidth}
        %\subfloat[Evolution continues] {
        %\input{figures/strip_narrowing/strip23.pdf_tex}
        %}%
    %\endgroup%
    \hfill
    \begingroup
    \def\svgwidth{0.21\textwidth}
        \captionsetup[subfigure]{width=0.28\textwidth}
        \subfloat[Length of leftmost (blue) segment becomes zero.]{%
          \makebox[0.28\textwidth]{\input{figures/strip_narrowing/strip24.pdf_tex}}
        }%
    \endgroup%

    \vspace{-1.2em}

    \begingroup
    \def\svgwidth{0.21\textwidth}
        \captionsetup[subfigure]{width=0.47\textwidth}
        \subfloat[Leftmost (zero length) segment is deleted.
                  The remaining two segments continue in the same direction.
                  Overall strip width has been reduced] {
        \input{figures/strip_narrowing/strip30.pdf_tex}
        %}%
    %\endgroup%
    %\begingroup
    \def\svgwidth{0.21\textwidth}
        %\captionsetup[subfigure]{width=0.22\textwidth}
        %\subfloat[] {
        \input{figures/strip_narrowing/strip31.pdf_tex}
        }%
    \endgroup%
    \hspace{1.5em}
    \begingroup
    \def\svgwidth{0.42\textwidth}
        \captionsetup[subfigure]{width=0.42\textwidth}
        \subfloat[Side view of strip narrowing gadget, with layers separated. The red lines denote the boundaries of the cross section.] {
        \input{figures/strip_narrowing/strip_layers.pdf_tex}
        }%
    \endgroup%
        \caption{Cross section evolution of a strip narrowing gadget \cite{strip_narrowing}.}
    \label{fig:strip_narrowing}
\end{figure}


\begin{definition}
\label{def:compatible}
Given two cross section intervals $\mathcal C$ and $\mathcal D$, such that $\mathcal C^F$ and $\mathcal D^I$ are equivalent,
we say that $\mathcal D$ is next-compatible with $\mathcal C$ and $\mathcal C$ is previous-compatible with $\mathcal D$.
Two cross sections $C = \langle s_1, s_2,\cdots s_n \rangle$ and $D = \langle r_1, r_2,\cdots r_m \rangle$ are equivalent
if and only if $C$ and $D$ correspond to the same sequence of segments after the deletion of all zero length segments.
\end{definition}

\begin{definition}
\label{def:cross_section_sequence}
A cross section sequence is a sequence is an ordered list of cross section intervals $ \langle \mathcal C_1, \mathcal C_2,\cdots \mathcal C_n \rangle$,
such that $C_{i}$ is next-compatible with $C_{i-1}$ for all $i\in [n-1]$.
This is equivalent to stating that $C_{i}$ is previous-compatible with $C_{i+1}$ for all $i\in [n-1]$.
Note that wedo not care about the directions of the segments.
\end{definition}

We will represent our full construction as a valid \emph{cross section sequence}.
Given a sross section sequence $\langle \mathcal C_1, \mathcal C_2,\cdots \mathcal C_n \rangle$,
the transition from $\mathcal C_i$ to $\mathcal C_{i+1}$ corresponds to the deletion of one or more
length zero segments from $\mathcal C_i$, and the addition of one or more zero length segments to obtain $\mathcal C_{i+1}$.
One simple example is shown in Figure~\ref{fig:strip_narrowing}.


\subsection{Evolution Corresponds to Flat Paper}
\label{sec:flat}

In this section we wil demonstrate that the folding formed by cross section evolution is realizable from a sheet of flat paper.
We note here that our construction may still result in self intersections.

We consider a cross section sequence $\langle \mathcal C_1, \mathcal C_2,\cdots \mathcal C_n \rangle$,
where each cross section interval $\mathcal C_i$ has evolution time $T_i$.
Say that the length ogf each cross section in the sequence is $X$.
We will denote a strip of paper of size $X\times L$ as an $L$-strip.

First, we will show that each $\mathcal C_i$ corresponds to a $T_i$-strip.
We will then use Theorem~\ref{def:cross_section_sequence} to attach the sequence of $T_i$-strips,
to form a complete $X\times T$ sheet of paper, where $T = \sum T_i$.

\subsubsection{Cross Section Interval Forms a Strip from a Gluing of Trapezoids}
\label{sec:cross_section_interval_strip}

In this section, we focus on a single cross section interval $\mathcal C$ with segments $\langle s_1, s_2,\cdots s_n \rangle$ evolving over time $T$.
First consider the surface traced out by an individual segment $s_i$.
Since the endpoints of $s_i$ move in a straight line, each segment traces \todo{What is a trace?} a trapezoid.

\begin{figure}[htb]
\graphicspath{{./figures/}}
    \centering
    \subfloat[] {
        \def\svgwidth{0.5\textwidth}
        \input{figures/trapezoid.pdf_tex}
        \label{fig:segment_trapezoid}
    }
    \hspace{-9em}
    \subfloat[] {
        \def\svgwidth{0.7\textwidth}
        \input{figures/trapezoid_angles.pdf_tex}
        \label{fig:trapezoid_angles}
    }
    \caption{}
    \label{fig:trapezoid}
\end{figure}

\begin{definition}
\label{def:trapezoid}
The surface traced out by a segment $s$ is a trapezoid $Z_s$ (Figure~\ref{fig:segment_trapezoid}).
Specifically, if $L,R$ are the initial left and right endpoints of $s$, and $L',R'$ are the final endpoints,
$Z_s = LL'R'R$ is the corresponding trapezoid (Figure~\ref{fig:segment_trapezoid}).
\end{definition}

\todo[inline]{Define gluing}

\begin{lemma}
\label{lem:trapezoid_gluing_parallel}
Consider a joint $J$ with segments $l$ and $r$, which has zero orthogonal joint velocity (i.e. $\vec v_l^\shortparallel = \vec v_r^\shortparallel = 0$.
The gluing of trapezoids $Z_l$ and $Z_r$ along the joint trajectory $\mathcal T_J$ is isometric to a larger trapezoid.
This joint trajectory is actually nothing but a crease in the folded state.
\end{lemma}

\begin{lemma}
\label{lem:trapezoid_gluing}

\end{lemma}


\begin{theorem}
\label{thm:cross_section_interval_strip}
The resulting gluing from any given cross section interval $\mathcal C$ with evolution time $T$ is isometric to a $T$-strip.
\end{theorem}


\subsubsection{Gluing the Interval Strips Together}
\label{sec:interval_strip_gluing}



