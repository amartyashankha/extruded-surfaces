\subsection{Evolution Corresponds to Flat Paper}
\label{sec:flat}

In this section we wil demonstrate that the folding formed by cross section evolution is realizable from a sheet of flat paper.
We note here that our construction may still result in self intersections.

We consider a cross section sequence $\langle \mathcal C_1, \mathcal C_2,\cdots \mathcal C_n \rangle$,
where each cross section interval $\mathcal C_i$ has evolution time $T_i$.
Say that the length ogf each cross section in the sequence is $X$.
We will denote a strip of paper of size $X\times L$ as an $L$-strip.

First, we will show that each $\mathcal C_i$ corresponds to a $T_i$-strip.
We will then use Theorem~\ref{def:cross_section_sequence} to attach the sequence of $T_i$-strips,
to form a complete $X\times T$ sheet of paper, where $T = \sum T_i$.

\subsubsection{Cross Section Interval Forms a Strip from a Gluing of Trapezoids}
\label{sec:cross_section_interval_strip}

In this section, we focus on a single cross section interval $\mathcal C$ with segments $\langle s_1, s_2,\cdots s_n \rangle$ evolving over time $T$.
First consider the surface traced out by an individual segment $s_i$.
Since the endpoints of $s_i$ move in a straight line, each segment traces \todo{What is a trace?} a trapezoid.

\begin{figure}[htb]
\graphicspath{{./figures/}}
    \centering
    \subfloat[] {
        \def\svgwidth{0.5\textwidth}
        \input{figures/trapezoid.pdf_tex}
        \label{fig:segment_trapezoid}
    }
    \hspace{-9em}
    \subfloat[] {
        \def\svgwidth{0.7\textwidth}
        \input{figures/trapezoid_angles.pdf_tex}
        \label{fig:trapezoid_angles}
    }
    \caption{}
    \label{fig:trapezoid}
\end{figure}

\begin{definition}
\label{def:trapezoid}
The surface traced out by a segment $s$ is a trapezoid $Z_s$ (Figure~\ref{fig:segment_trapezoid}).
Specifically, if $L,R$ are the initial left and right endpoints of $s$, and $L',R'$ are the final endpoints,
$Z_s = LL'R'R$ is the corresponding trapezoid (Figure~\ref{fig:segment_trapezoid}).
\end{definition}

\todo[inline]{Define gluing}

\begin{lemma}
\label{lem:trapezoid_gluing_parallel}
Consider a joint $J$ with segments $l$ and $r$, which has zero orthogonal joint velocity (i.e. $\vec v_l^\shortparallel = \vec v_r^\shortparallel = 0$.
The gluing of trapezoids $Z_l$ and $Z_r$ along the joint trajectory $\mathcal T_J$ is isometric to a larger trapezoid.
This joint trajectory is actually nothing but a crease in the folded state.
\end{lemma}

\begin{lemma}
\label{lem:trapezoid_gluing}

\end{lemma}


\begin{theorem}
\label{thm:cross_section_interval_strip}
The resulting gluing from any given cross section interval $\mathcal C$ with evolution time $T$ is isometric to a $T$-strip.
\end{theorem}


\subsubsection{Gluing the Interval Strips Together}
\label{sec:interval_strip_gluing}


