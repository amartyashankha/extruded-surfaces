\subsection{Joints}
\label{sec:joints}

\begin{definition}
\label{def:joints}
A cross section with $n$ segments is also associated with a list of joints $ \langle J_1,\cdots J_{n-1} \rangle$,
where $J_i$ corresponds to the right endpoint of $s_i$ (same as left endpoint of $s_{i+1}$.
A particular joint $\vec J_i$ is associated with a left segment $s_i$, a right segment $s_{i+1}$, and a velocity $\vec J_v$.
\end{definition}

\begin{definition}
\label{def:joint_plane}
A joint plane is the plane that coincides with both segments $l$ and $r$ associated with a particular joint $J$.
\end{definition}

\begin{definition}
\label{def:joint_plane_velocity}
Consider a joint $J$ associated with segments $l$, $r$, and joint plane $\mathcal P$,
where $\vec v_l$ and $\vec v_r$ are the velocities of segments $l$ and $r$.
We define $\vec v_l^\shortparallel$ and $\vec v_l^\perp$ as the components of $\vec v_l$ coinciding with,
and orthogonal to $\mathcal P$ respectively.
Similarly, we define $\vec v_r^\shortparallel$ and $\vec v_r^\perp$, as the components of $\vec v_r$.
By Property~\ref{pro:orthogonal_velocity}, $\vec v_l^\shortparallel$ and $\vec v_r^\shortparallel$
have to be orthogonal to $\vec{\hat o_l}$ and $\vec{\hat o_r}$ respectively.
\end{definition}

\begin{restatable}{pro}{JointOrthogonalVelocity}
\label{pro:joint_orthogonal_velocity}
For a joint $J$ associated with segments $l$ and $r$, $\vec v_l^\perp = \vec v_r^\perp$.
As a corollary, $ \left\| \vec v_l^\shortparallel\right\| = \left\| \vec v_r^\shortparallel\right\|$.
\end{restatable}
