\subsection{Joints}
\label{sec:joints}

\begin{definition}
\label{def:joints}
A cross section with $n$ segments is also associated with a list of joints
$ \langle J_1,\cdots J_{n-1} \rangle$, where $J_i$ corresponds to the right endpoint of $s_i$.
The velocity of a joint $J_i$, is $\vec J_v = \vec v_i+\vec v_{i+1}$
(i.e. the sum of the velocities of the corresponding segments).
\end{definition}
\begin{definition}
\label{def:creases}
Note that the trajectory of the joints form a crease \ldots angle is angle between direction vectors.
Creases are also created when a segment changes direction\ldots angle is change in direction vector.
\end{definition}

\begin{definition}
\label{def:joint_plane}
A joint plane is the plane that coincides with both segments $l$ and $r$ associated with a particular joint $J$.
\end{definition}

\begin{definition}
\label{def:joint_plane_velocity}
Consider a joint $J$ associated with segments $l$ and $r$, and joint plane $\mathcal P$,
where $\vec v_l$ and $\vec v_r$ are the velocities of segments $l$ and $r$.
We define $\vec v_l^\shortparallel$ and $\vec v_l^\perp$ as the components of $\vec v_l$ coinciding with,
and orthogonal to the joint plane respectively.
Similarly, we define $\vec v_r^\shortparallel$ and $\vec v_r^\perp$, as the components of $\vec v_r$.
Further, note that $\vec v_l^\shortparallel$ and $\vec v_r^\shortparallel$
have to be orthogonal to $\vec o_l$ and $\vec o_r$ respectively.
\end{definition}

\begin{claim}
\label{clm:joint_orthogonal_velocity}
For a joint $J$ associated with segments $l$ and $r$, $\vec v_l^\perp = \vec v_r^\perp$.
\end{claim}

\begin{corollary}
\label{cor:joint_plane_velocity}
For a joint $J$ associated with segments $l$ and $r$,
$ \left\| \vec v_l^\shortparallel\right\| = \left\| \vec v_r^\shortparallel\right\|$.
\end{corollary}

\begin{claim}
\label{clm:valid_joint}
A joint $J$ corresponding to segments $s$ and $t$ is \emph{valid}
if the resulting evolution of the joint preserves distances.
\end{claim}
