\graphicspath{{./figures/strip_narrowing/}}
\begin{figure}[]
    \begingroup
    \def\svgwidth{0.23\textwidth}
        \captionsetup[subfigure]{width=0.22\textwidth}
        \subfloat[The trivial cross section.] {
        \input{figures/strip_narrowing/strip0.pdf_tex}
        }%
    \endgroup%
    \begingroup
    \def\svgwidth{0.23\textwidth}
        \captionsetup[subfigure]{width=0.48\textwidth}
        \subfloat[New zero length segment inserted at left endpoint. Existing segment reverses velocity.] {
        \input{figures/strip_narrowing/strip10.pdf_tex}
        %}%
    %\endgroup%
    %\begingroup
    \def\svgwidth{0.23\textwidth}
        %\captionsetup[subfigure]{width=0.22\textwidth}
        %\subfloat[] {
        \input{figures/strip_narrowing/strip11.pdf_tex}
        }%
    \endgroup%
    \begingroup
    \def\svgwidth{0.23\textwidth}
        \captionsetup[subfigure]{width=0.24\textwidth}
        \subfloat[Evolution continues] {
        \input{figures/strip_narrowing/strip12.pdf_tex}
        }%
    \endgroup%
    \vspace{-1.2em}
    \begingroup
    \def\svgwidth{0.23\textwidth}
        \captionsetup[subfigure]{width=0.67\textwidth}
        \subfloat[New zero length segment (red) inserted between two existing segments
                  with the same velocity as the rightmost segment.
                  Existing segments retain their velocites.] {
        \input{figures/strip_narrowing/strip20.pdf_tex}
        %}%
    %\endgroup%
    %\begingroup
    \def\svgwidth{0.23\textwidth}
        %\captionsetup[subfigure]{width=0.22\textwidth}
        %\subfloat[] {
        \input{figures/strip_narrowing/strip21.pdf_tex}
        %}%
    %\endgroup%
    %\begingroup
    \def\svgwidth{0.23\textwidth}
        %\captionsetup[subfigure]{width=0.22\textwidth}
        %\subfloat[] {
        \input{figures/strip_narrowing/strip22.pdf_tex}
        }%
    \endgroup%
    %\begingroup
    %\def\svgwidth{0.23\textwidth}
        %\captionsetup[subfigure]{width=0.22\textwidth}
        %\subfloat[Evolution continues] {
        %\input{figures/strip_narrowing/strip23.pdf_tex}
        %}%
    %\endgroup%
    \begingroup
    \def\svgwidth{0.23\textwidth}
        \captionsetup[subfigure]{width=0.28\textwidth}
        \subfloat[Length of leftmost (blue) segment becomes zero.] {
        \input{figures/strip_narrowing/strip24.pdf_tex}
        }%
    \endgroup%
    \vspace{-1.2em}
    \begingroup
    \def\svgwidth{0.23\textwidth}
        \captionsetup[subfigure]{width=0.47\textwidth}
        \subfloat[Leftmost (zero length) segment is deleted.
                  The remaining two segments continue in the same direction.
                  Overall strip width has been reduced] {
        \input{figures/strip_narrowing/strip30.pdf_tex}
        %}%
    %\endgroup%
    %\begingroup
    \def\svgwidth{0.23\textwidth}
        %\captionsetup[subfigure]{width=0.22\textwidth}
        %\subfloat[] {
        \input{figures/strip_narrowing/strip31.pdf_tex}
        }%
    \endgroup%
    \hspace{1.5em}
    \begingroup
    \def\svgwidth{0.42\textwidth}
        \captionsetup[subfigure]{width=0.42\textwidth}
        \subfloat[Side view of strip narrowing gadget, with layers separated. The red lines denote the boundaries of the cross section.] {
        \input{figures/strip_narrowing/strip_layers.pdf_tex}
        }%
    \endgroup%
        \caption{Cross section evolution of a strip narrowing gadget \cite{strip_narrowing}.}
    \label{fig:strip_narrowing}
\end{figure}
