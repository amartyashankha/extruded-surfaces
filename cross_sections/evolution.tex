\subsection{Time Travel}
\label{sec:time_travel}

In the process of time travel, all the nodes on a segment $s$, except the
\emph{joint nodes} move with velocity $\vec{\hat v_s}$ (orthogonal to $\vec{\hat
o_s}$). For example, if we allow a single segment of length $X$ to evolve for
time $T$, it will result in a $X\times T$ strip of paper.

The joint node velocity $\vec J_v$ may have a component along a correspoding
segment $s$ (along $\vec{\hat o_s}$). As a result, the lengths of segments may
change (Figure~\ref{fig:trapezoid_angles}). This can be visualized as movement
of the corresponding joint along one of the segments.

\begin{definition}
\label{def:segment_length}
For ever segment $s$ in a cross section $C$, we associate a left pace $L_s$,
which indicates the rate at which $s$ shrinks from its left endpoint.
Similarly, we define a right pace $R_s$ grows from its right endpoint.
Note that both these quantities can be negative.
\end{definition}
After time $T$, the length of a segment changes by $T\cdot(R_s-L_s)$.
The length of a segment is not allowed to become negative.
For a segment $s_i$ with left joint $J^L$ and right joint $J^R$, we obtain the relations
\begin{align*}
\vec J_v^L-\vec{\hat v_{s_i}} = L_{s_i}\cdot \vec{\hat o_{s_i}} && \vec J_v^R-\vec{\hat v_{s_i}} = R_{s_i}\cdot \vec{\hat o_{s_i}}
\end{align*}

\begin{proposition}
\label{prop:valid_joint}
A joint $J$ corresponding to segments $s$ and $t$ is \emph{valid} if and only if the evolution
resulting from the velocities $\vec{\hat v_l}$, $\vec{\hat v_r}$, and $\vec J_v$ preserves distances between nodes.
\end{proposition}
\vspace{-1pc}
\begin{restatable}{pro}{LeftRightPace}
\label{pro:left_right_pace}
The right pace of $s_{i}$ is equal to the left pace of $s_{i+1}$.
This is to preserve the overall length of the cross section, and the distance between any two nodes.
Furthermore, the left pace of the first segment, and the right pace of the last segment should be zero, i.e. $L_0 = R_n = 0$
This ensures that the total length of the cross section does not change.
\end{restatable}

The movement of a joint increases the length of one of its associated segments,
and decreases the length of the other segment by the same amount (this ensures that the total length is preserved).
This puts some constraints on the possible velocities of adjacent segments.

Consider a joint $\vec J_i$ corresponding to segments $l = s_i$ and $r = s_{i+1}$, at time $t=0$.
Henceforth, we will refer to $L_r$ as $L$, and $R_l$ as $R$.
Without loss of genenerality, we assume that $L<0$.
At a later time $t$, let the new joint position be $\vec J'_i$.
We define nodes $a$ and $b$ corresponding to $\vec J_i$ and $\vec J_{i+1}$ respectively.
We also define the initial and final positions of $a$ as $\vec a$,
and $\vec a'$, and similarly for $b$, we define $\vec b$ and $\vec b'$.
Let $d$ be the \emph{separation} between nodes $a$ and $b$.
This setup is shown in Figure~\ref{fig:joint_plane_velocity}.

%\begin{figure}[htpb]
\begin{wrapfigure}[8]{r}{0.4\textwidth}
%\vspace{-2.2em}
\graphicspath{{./notebooks/}}
    \centering
    \def\svgwidth{\linewidth}
    \input{notebooks/joint_plane_velocity.pdf_tex}
    \caption{A joint with segments $l$ and $r$.
    The trajectory of the joint is shown in orange.
    The trajectories of $a$ and $b$ are shown in blue.
    The green arrows indicate $\vec v^\shortparallel$}
    \label{fig:joint_plane_velocity}
\end{wrapfigure}
%\end{figure}


First, note that $\vec a, \vec b$ lie on the segment $l$, and $\vec a', \vec b'$ lie on the segment $r$,
which implies that $\vec b-\vec a = d\cdot \vec{\hat o_l}$, and $\vec b'-\vec a' = d\cdot \vec{\hat o_r}$.
\begin{align*}
\vec b' - \vec a' &= (\vec b + t\cdot \vec{\hat v_r}^\shortparallel) - (\vec a + t\cdot \vec{\hat v_l}^\shortparallel)\\
\implies \vec b' - \vec a' &= (\vec b - \vec a) + t\cdot(\vec{\hat v_r}^\shortparallel - \vec{\hat v_l}^\shortparallel)\\
\implies d\cdot \vec{\hat o_r} &= d\cdot \vec{\hat o_l} + t\cdot(\vec{\hat v_r}^\shortparallel - \vec{\hat v_l}^\shortparallel)\\
\implies \vec{\hat v_r}^\shortparallel - \vec{\hat v_l}^\shortparallel &= \frac{d}{t}\cdot(\vec{\hat o_r}-\vec{\hat o_l})
\\&= -R\cdot(\vec{\hat o_r}-\vec{\hat o_l}) = -L\cdot(\vec{\hat o_r}-\vec{\hat o_l})
\end{align*}
This is only possible if $\vec{\hat o_l}\times \vec{\hat o_r}$ is oriented opposite to $\vec{\hat v_l}^\shortparallel\times \vec{\hat v_r}^\shortparallel$.

\begin{restatable}{pro}{SegmentOrientation}
\label{pro:SegmentOrientation}
Given two adjacent segments $l$ and $r$ in a cross section $C$, the vector
$\vec{\hat o_l}\times \vec{\hat o_r}$ must be oriented opposite to $\vec{\hat v_l}^\shortparallel\times \vec{\hat v_r}^\shortparallel$.
\end{restatable}

If the angle between the segments (between $\vec{\hat o_l}$ and $\vec{\hat o_r}$) is $\theta$,
the magnitude of $\vec{\hat o_r}-\vec{\hat o_l}$ is $\sqrt{2-2\cos(\theta)}$,
and $ \left\| \vec{\hat v_r}^\shortparallel-\vec{\hat v_l}^\shortparallel\right\| = v\cdot\sqrt{2-2\cos(\pi-\theta)}$.
Here, $v$ is magnitude of the plane velocity of $J_i$. Given $\phi = \theta/2$, we get --
\begin{align*}
-L = -R &= \frac dt = \vec{\hat v_l} - \vec{\hat o_l}\frac{ \left\| \vec{\hat v_r}^\shortparallel
- \vec{\hat v_l}^\shortparallel\right\|}{ \left\| \vec{\hat o_r}-\vec{\hat o_l}\right\|}
&= v\cdot\frac{\sqrt{\sin^2\left( {\pi/2-\theta/2}\right)}}{\sqrt{\sin^2{\theta/2}}}
= v\cdot\cot\left( \phi \right)
\end{align*}

\begin{restatable}{pro}{JointVelocity}
\label{pro:joint_velocity}
The velocity of a joint $J$ associated with segments $l$ and $r$, is a constant vector
$\vec J_v
%= \vec{\hat v_l} - \vec{\hat o_l}\frac{ \left\| \vec{\hat v_r}^\shortparallel
%- \vec{\hat v_l}^\shortparallel\right\|}{ \left\| \vec{\hat o_r}-\vec{\hat o_l}\right\|}
= \vec{\hat v_l} - \left\| \vec{\hat v_l}^\shortparallel\right\|\cdot\vec{\hat o_l}\cdot\cot\left( \theta/2 \right)$,
where $\theta$ is the angle between $\vec{\hat o_l}$ and $\vec{\hat o_r}$.
\end{restatable}

%\todo[inline]{
%Note that time evolution is reversible.
%}
