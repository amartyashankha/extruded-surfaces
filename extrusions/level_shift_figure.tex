\graphicspath{{./figures/level_shift/}}
\begin{figure}[!htb]
    \centering
    \begingroup
        \def\svgwidth{0.26\textwidth}
        \captionsetup[subfigure]{width=0.26\textwidth}
        \subfloat[Higher initial level. Top segment moves down.]%
        {\input{./figures/level_shift/column0.pdf_tex}}
    \endgroup
    \subfloat[Two segments created. New segments are aligned with top segment, and move down. Vertical segments move inwards.] {
        \def\svgwidth{0.25\textwidth}
        \input{./figures/level_shift/column1.pdf_tex}
        \def\svgwidth{0.26\textwidth}
        \input{./figures/level_shift/column2.pdf_tex}
    }%
    \def\svgwidth{0.26\textwidth}
    \subfloat[Two more segments created. Vertical segments reverse direction.] {
        \input{./figures/level_shift/column3.pdf_tex}
    %}%
        \def\svgwidth{0.28\textwidth}
    %\subfloat[] {
        \input{./figures/level_shift/column4.pdf_tex}
    %}%
        \def\svgwidth{0.30\textwidth}
    %\subfloat[] {
        \input{./figures/level_shift/column5.pdf_tex}
    }%

    \subfloat[Level shift completed with four new horizontal segments.] {
        \def\svgwidth{0.31\textwidth}
        \input{./figures/level_shift/column6.pdf_tex}
        \def\svgwidth{0.36\textwidth}
        \input{./figures/level_shift/column7.pdf_tex}
    }%
    \begingroup
        \def\svgwidth{0.27\textwidth}
        \captionsetup[subfigure]{width=0.25\textwidth}
        \subfloat[Flat folded state.]
        {\input{./figures/level_shift/column8.pdf_tex}}%
    \endgroup%
    \caption{Level Shifting Gadget. The separation along the $Y$ direction serves to illustrate the layering.}
    \label{fig:level_shift}
\end{figure}
