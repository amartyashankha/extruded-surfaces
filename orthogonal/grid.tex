\subsection{Multiple Column Extrusions form a Grid Extrusion}
\label{sec:grid_extrusion}

Now we consider mutiple column extrusions evolving in parallel.
Henceforth, we will refer to the evolution of column cross sections along the $y$-axis
(the ''1''s in Equation~\ref{eq:column_decomposition} as \emph{horizontal evoluton}.
Meanwhile, vertical transition will refer to level shifting evolution in the $x-z$ plane.
Let $\mathcal C^{(i)}$ be a valid cross section evolution corresponding
to the $i^{th}$ column in the grid extrusion (as defined in Section~\ref{sec:column_extrusion}).
As before, for simplicity, we will assume that the minimum height in each column is zero $\left( i.e.\ \min_j\left\{ E_{i,j}\right\} = 0 \right)$.

\graphicspath{{./figures/up_down/}}%
\begin{figure}[ht!]%
    \centering
    \begingroup%
        \captionsetup[subfigure]{width=0.18\textwidth}
        \subfloat[Column starts at initial level $H_i$] {
            \def\svgwidth{0.22\textwidth}
            \input{./figures/up_down/up_down0.pdf_tex}%
            \label{fig:up_down0}
        }%
    \endgroup
    \begingroup%
        \captionsetup[subfigure]{width=0.44\textwidth}
        \subfloat[Two segments created. New segments are aligned with top segment, and move down.] {
            \def\svgwidth{0.205\textwidth}
            \input{./figures/up_down/up_down1.pdf_tex}%
        %}%
        %\subfloat[] {
            \def\svgwidth{0.215\textwidth}
            \input{./figures/up_down/up_down2.pdf_tex}%
            \label{fig:up_down1}
        }%
    \endgroup
    \begingroup%
        \captionsetup[subfigure]{width=0.2\textwidth}
        \subfloat[Segments reverse direction] {
            \def\svgwidth{0.22\textwidth}
            \input{./figures/up_down/up_down3.pdf_tex}%
            \label{fig:up_down2}
        }%
    \endgroup

    \begingroup%
        \captionsetup[subfigure]{width=0.55\textwidth}
        \subfloat[Cross section returns to initial state, and continues horizontal evolution.] {
            \def\svgwidth{0.28\textwidth}
            \input{./figures/up_down/up_down4.pdf_tex}%
        %}%
        %\subfloat[] {
            \def\svgwidth{0.31\textwidth}
            \input{./figures/up_down/up_down5.pdf_tex}%
            \label{fig:up_down3}
        }%
    \endgroup
    \begingroup%
        \captionsetup[subfigure]{width=0.27\textwidth}
        \subfloat[Flat folded state] {
            \def\svgwidth{0.27\textwidth}
            \input{./figures/up_down/up_down6.pdf_tex}%
            \label{fig:up_down4}
        }%
    \endgroup
    \caption{Up-down gadget. The separation along the $Y$ direction illustrates the layering. The red line denotes the boundary of the cross section.}
    \label{fig:up_down}
\end{figure}%
%

We consider the parallel evolution of each column extrusion $\mathcal C^{(i)}$.
\begin{itemize}
    \item During the horizontal evolution of row $j$, each $\mathcal C^{(i)}$ evolves along the positive $y$ direction for time $1$ at height $E_{i,j}$.
    \item During the vertical transition from row $j$ to $j+1$, each $\mathcal C^{(i)}$ evolves for time $ D_{ij} = \left| E_{i,j+1}-E_{i,j}\right|$.
\end{itemize}
Note that the vertical transition times are different for each $\mathcal C^{(i)}$.
Since we want to glue the sequence of $\left\{ \mathcal C^{(i)}\right\}$s, our constructions must have equal transition times.

\begin{definition}
\label{def:slowest_column}
We define the common transition time from row $j$ to row $j+1$ as
$$D_j = \max_j\left\{ D_{ij}\right\} = \max_j\left\{ \left| E_{i,j+1}-E_{i,j}\right|\right\}$$
\end{definition}

So, the slowest column dictates the transition time, and the faster columns have to \emph{stall} for additional time.
To achieve this, we define an \emph{up-down} gadget, which is very similar to our original level shifting gadget.
This up-down gadget (Figure~\ref{fig:up_down}) evolves for $2\varepsilon$ time, but the height of the corresponding column remains unchanged.
This gadget starts at a height $h$, and for the first $\varepsilon$ time interval (Figure~\ref{fig:up_down1}),
evolves exactly the same way as the down-shift gadget (Figure~\ref{fig:level_shift1}).
For the second $\varepsilon$ time interval, the cross section evolves in reverse (Figure~\ref{fig:up_down2}),
back to it's original state (Figure~\ref{fig:up_down3}).

So, the vertical transition of $\mathcal C^{(i)}$ needs to use a total of $\left( D_j - D_{i,j}\right)/\left( 2\varepsilon\right)$
up-down gadgets (each gadget \emph{stalls} for $2\varepsilon$ time).
%Notice that this gadget requires at least one accordion segment. In fact, by Lemma~\ref{lem:accordion_even}, it requires two segments.
%Given the worst case scenario, where $E_{i,j}=E_{i,j+1}$ is the max height in column $i$,
We obtain the following primitive, as a consequence of Theorem~\ref{thm:column_extrusion}.

\begin{proposition}
\label{prop:accordion_layers}
%The number of accordion layers for column $i$ at its minium height is
%$L_i = 2 + \left( \max_j\left\{ E_{i,j}\right\} - \min_j\left\{ E_{i,j}\right\} \right)/\varepsilon$.
By Theorem~\ref{thm:column_extrusion}, the extrusion for column $i$ is constructed from a paper strip with size
$$\left( 1 + 2\cdot\max_j\left\{ E_{ij}\right\}\right) {\sc X}  m + \sum\limits^{m-1}_{j=1} D_j $$
\end{proposition}
