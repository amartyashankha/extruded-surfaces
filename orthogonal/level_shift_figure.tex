\graphicspath{{./figures/level_shift/}}%
\begin{figure}[t!]%
    \centering
    \begingroup%
        \def\svgwidth{0.25\textwidth}
        \captionsetup[subfigure]{width=0.25\textwidth}
        \subfloat[Higher initial level. Top segment moves down.]{%
            \input{./figures/level_shift/column0.pdf_tex}
            \label{fig:level_shift0}
        }%
    \endgroup
    \subfloat[Two segments created. New segments are aligned with top segment, and move down. Vertical segments move inwards.] {%
        \def\svgwidth{0.24\textwidth}
        \input{./figures/level_shift/column1.pdf_tex}%
        \def\svgwidth{0.25\textwidth}
        \input{./figures/level_shift/column2.pdf_tex}%
        \label{fig:level_shift1}
    }%
    \def\svgwidth{0.25\textwidth}
    \subfloat[Two more segments created. Vertical segments reverse direction.] {%
        \input{./figures/level_shift/column3.pdf_tex}%
    %}%
        \def\svgwidth{0.27\textwidth}
    %\subfloat[] {
        \input{./figures/level_shift/column4.pdf_tex}%
    %}%
        \def\svgwidth{0.29\textwidth}
    %\subfloat[] {
        \input{./figures/level_shift/column5.pdf_tex}%
        \label{fig:level_shift2}
    }%

    \subfloat[Level shift completed with four new horizontal segments.] {%
        \def\svgwidth{0.29\textwidth}
        \input{./figures/level_shift/column6.pdf_tex}%
        \def\svgwidth{0.34\textwidth}
        \input{./figures/level_shift/column7.pdf_tex}%
        \label{fig:level_shift3}
    }%
    \begingroup
        \def\svgwidth{0.27\textwidth}
        \captionsetup[subfigure]{width=0.24\textwidth}
        \subfloat[Flat folded state.]{%
            \input{./figures/level_shift/column8.pdf_tex}%
            \label{fig:level_shift4}
        }%
    \endgroup%
    \caption{Level shifting gadget. The separation along the $Y$ direction illustrates the layering. The red line denotes the boundary of the cross section.}
    \label{fig:level_shift}
\end{figure}%
