\graphicspath{{./figures/up_down/}}%
\begin{figure}[ht!]%
    \centering
    \begingroup%
        \captionsetup[subfigure]{width=0.18\textwidth}
        \subfloat[Column starts at initial level $H_i$]{
            \def\svgscale{0.1}
            \input{./figures/up_down/up_down0.pdf_tex}%
            \label{fig:up_down0}
        }%
    \endgroup
    \begingroup%
        %\captionsetup[subfigure]{width=0.44\textwidth}
        \subfloat[Two segments created. New segments are aligned with top segment, and move down.] {
            \def\svgscale{0.1}
            \input{./figures/up_down/up_down1.pdf_tex}%
        %}%
        %\subfloat[] {
            \def\svgscale{0.1}
            \input{./figures/up_down/up_down2.pdf_tex}%
            \label{fig:up_down1}
        }%
    \endgroup
    \begingroup%
        \captionsetup[subfigure]{width=0.2\textwidth}
        \subfloat[Segments reverse direction] {
            \def\svgscale{0.1}
            \input{./figures/up_down/up_down3.pdf_tex}%
            \label{fig:up_down2}
        }%
    \endgroup

    \begingroup%
        %\captionsetup[subfigure]{width=0.55\textwidth}
        \subfloat[Cross section returns to initial state, and continues horizontal evolution.] {
            \def\svgscale{0.1}
            \input{./figures/up_down/up_down4.pdf_tex}%
        %}%
        %\subfloat[] {
            \def\svgscale{0.1}
            \input{./figures/up_down/up_down5.pdf_tex}%
            \label{fig:up_down3}
        }%
    \endgroup
    \begingroup%
        \captionsetup[subfigure]{width=0.27\textwidth}
        \subfloat[Flat folded state] {
            \def\svgscale{0.1}
            \input{./figures/up_down/up_down6.pdf_tex}%
            \label{fig:up_down4}
        }%
    \endgroup
    \caption{Up-down gadget. The separation along the $Y$ direction illustrates the layering. The red line denotes the boundary of the cross section.}
    \label{fig:up_down}
\end{figure}%
