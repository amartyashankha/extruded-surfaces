\graphicspath{{./figures/up_down/}}%
\begin{figure}[ht!]%
    \centering
    \subfloat[] {
        \def\svgwidth{0.22\textwidth}
        \input{./figures/up_down/up_down0.pdf_tex}%
        \label{fig:up_down0}
    }%
    \subfloat[] {
        \def\svgwidth{0.195\textwidth}
        \input{./figures/up_down/up_down1.pdf_tex}%
        \label{fig:up_down1}
    %}%
    %\subfloat[] {
        \def\svgwidth{0.20\textwidth}
        \input{./figures/up_down/up_down2.pdf_tex}%
        \label{fig:up_down2}
    }%
    \subfloat[] {
        \def\svgwidth{0.21\textwidth}
        \input{./figures/up_down/up_down3.pdf_tex}%
        \label{fig:up_down3}
    }%

    \subfloat[] {
        \def\svgwidth{0.25\textwidth}
        \input{./figures/up_down/up_down4.pdf_tex}%
        \label{fig:up_down4}
    }%
    \subfloat[] {
        \def\svgwidth{0.28\textwidth}
        \input{./figures/up_down/up_down5.pdf_tex}%
        \label{fig:up_down5}
    }%
    \subfloat[] {
        \def\svgwidth{0.24\textwidth}
        \input{./figures/up_down/up_down6.pdf_tex}%
        \label{fig:up_down6}
    }%
    \caption{Level Shifting Gadget. The separation along the $Y$ direction illustrates the layering. The red line denotes the boundary of the cross section.}
    \label{fig:up_down}
\end{figure}%
