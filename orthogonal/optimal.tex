\subsection{Worst-Case Efficiency}
\label{sec:optimality}

Consider an $n\times n$ orthogonal terrain where the highest points along $x$-axis are $M = \max_j E_{ij}$,
and the maximum deltas along the $y$-axis are $D_j = \max_i{\|E_{i,j}-E_{i,j+1}\|} = M$.
For example, one such ``worst-case'' orthogonal terrain is the following
checkerboard alternating between the minimum and maximum heights:
$$
E_{ij}=
\begin{cases}
M & \textrm{ if $(i+j)$ is even,}\\
0 & \textrm{ if $(i+j)$ is odd.}
\end{cases}
$$

We assume that the top faces of the terrain form axis-aligned squares in the unfolded state,
The longest line formed along either axis is the total length of the top surfaces plus the sum of the height changes across the axis, which gives
$$L = n + \sum^{n}_{i=1} M = n + (n-1)\cdot M.$$
We conclude that the minimum required size of a folding is $L\times L$.

From Theorem~\ref{thm:grid_extrusion}, we know that the terrain can be folded from an $X\times Y$ strip of paper,
where $X = n + 2(n-1)\cdot M + o(1)$ and $Y = n + (n-1)\cdot M$.
Since $Y = L$, and $X < 2L$ (assuming an appropriately small value of $\varepsilon$),
our construction results in a folding that is within a factor two of the optimal paper usage for this terrain, under the aforementioned assumption.

%\begin{claim}
%The $x$-axis length of the strip of paper required to fold this shape can be made arbitrarily close to $X$.
%\end{claim}
%\begin{claim}
%The $y$-axis length of the strip of paper required to fold this shape will be exactly $Y$.
%\end{claim}

%First, we pick an $\epsilon$, such that $2\varepsilon$ divides all extrusion heights.
%The construction will require paper of size $X'\times Y$, where $X' = X + 2\epsilon(n-1)$.

%There are two components to the construction
%\begin{itemize}
	%\item $n$ strips parallel to the $y$-axis. Each of these strips will fold to the corresponding strip in the extruded graph. The total area of these strips will be $X\times Y$.
    %\item $n-1$ Intermediate strips, each of size $2\epsilon\times Y$ to connect the main strips together.
%\end{itemize}
%The total area is therefore $X\times Y + (n-1)2\epsilon\times Y = X'\times Y$. This can of course be made arbitrarily close to $X\times Y$.

