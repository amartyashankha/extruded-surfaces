\subsection{Optimality}
\label{sec:optimality}

Under some suitable assumptions, our construction can be made $2+\varepsilon$ optimal,
for arbitrarily small $\varepsilon$.

We define the maximum deltas along the $y$-axis as $D_j = \max_i{\|E_{i,j}-E_{i,j+1}\|}$,
and let $Y = n + \sum_{i=1}^{n-1} D_j$
We also define the lowest and highest points along $x$-axis as
$L_i = \max_j E_{ij}$ and $H_i = \max_j E_{ij}$, and let
$$X = n + \sum\limits_{i=1}^n \left[(H_i-\min(L_i,L_{i+1})) + (H_i-\min(L_i,L_{i-1}))\right]$$

The terms in the summation account for the total length of all the necessary worst case vertical walls,
and the $n$ is for the top faces.

\begin{claim}
The $x$-axis length of the strip of paper required to fold this shape can be made arbitrarily close to $X$.
\end{claim}
\begin{claim}
The $y$-axis length of the strip of paper required to fold this shape will be exactly $Y$.
\end{claim}

First, we pick an $\epsilon$, such that $2\varepsilon$ divides all extrusion heights.
The construction will require paper of size $X'\times Y$, where $X' = X + 2\epsilon(n-1)$.

There are two components to the construction
\begin{itemize}
	\item $n$ strips parallel to the $y$-axis. Each of these strips will fold to the corresponding strip in the extruded graph. The total area of these strips will be $X\times Y$.
    \item $n-1$ Intermediate strips, each of size $2\epsilon\times Y$ to connect the main strips together.
\end{itemize}
The total area is therefore $X\times Y + (n-1)2\epsilon\times Y = X'\times Y$. This can of course be made arbitrarily close to $X\times Y$.

