\subsection{Construction of Column Extrusion by Level Shifting}
\label{sec:column_extrusion}

First, we consider a single column (Figure~\ref{fig:column_extrusion})
of the orthogonal terrain $\left\{ E_{i,1}, E_{i,2}, \cdots, E_{i,n} \right\}$.
We denote the column extrusion heights as $\left\{ H_1, H_2,\cdots H_n \right\}$, where $H_j = E_{i,j}$.
Consider the decomposition of $T$ into the following time intervals:% (parallel to the $x$-axis):
\begin{align}
    \label{eq:column_decomposition}
T = 1 + D_1  +  1 + D_2  +  1 + D_3  +\cdots\cdots +  1 + D_{m-1}  +  1.
\end{align}
Here, the time corresponding to the $i$th 1 is realized as the surface at height $H_i$,
and the time corresponding to $D_i = \left| H_i-H_{i+1}\right|$ is the transition between $H_i$ and $H_{i+1}$.

\graphicspath{{./figures/}}
\begin{figure}[htb]
    \def\svgwidth{1.0\textwidth}
    \input{./figures/level_shift_layers.pdf_tex}%
    \caption{
    Cross section change from level $i$ to $i+1$. The \emph{accordion segments} are separated for illustration.
    In reality, the accordion is folded flat. Zero distances are marked by $\phi$.
    The velocities of horizontal and vertical segments are shown by purple and green arrows respectively.
    }
    \label{fig:level_shift_layers}
\end{figure}

\graphicspath{{./figures/level_shift/}}%
\begin{figure}[t!]%
    \centering
    \begingroup%
        \def\svgwidth{0.25\textwidth}
        \captionsetup[subfigure]{width=0.25\textwidth}
        \subfloat[Higher initial level. Top segment moves down.]{%
            \input{./figures/level_shift/column0.pdf_tex}
            \label{fig:level_shift0}
        }%
    \endgroup
    \subfloat[Two segments created. New segments are aligned with top segment, and move down. Vertical segments move inwards.] {%
        \def\svgwidth{0.24\textwidth}
        \input{./figures/level_shift/column1.pdf_tex}%
        \def\svgwidth{0.25\textwidth}
        \input{./figures/level_shift/column2.pdf_tex}%
        \label{fig:level_shift1}
    }%
    \def\svgwidth{0.25\textwidth}
    \subfloat[Two more segments created. Vertical segments reverse direction.] {%
        \input{./figures/level_shift/column3.pdf_tex}%
    %}%
        \def\svgwidth{0.27\textwidth}
    %\subfloat[] {
        \input{./figures/level_shift/column4.pdf_tex}%
    %}%
        \def\svgwidth{0.29\textwidth}
    %\subfloat[] {
        \input{./figures/level_shift/column5.pdf_tex}%
        \label{fig:level_shift2}
    }%

    \subfloat[Level shift completed with four new horizontal segments.] {%
        \def\svgwidth{0.30\textwidth}
        \input{./figures/level_shift/column6.pdf_tex}%
        \def\svgwidth{0.35\textwidth}
        \input{./figures/level_shift/column7.pdf_tex}%
        \label{fig:level_shift3}
    }%
    \begingroup
        \def\svgwidth{0.25\textwidth}
        \captionsetup[subfigure]{width=0.24\textwidth}
        \subfloat[Flat folded state.]{%
            \input{./figures/level_shift/column8.pdf_tex}%
            \label{fig:level_shift4}
        }%
    \endgroup%
    \caption{Level Shifting Gadget. The separation along the $Y$ direction illustrates the layering. The red line denotes the boundary of the cross section.}
    \label{fig:level_shift}
\end{figure}%
%

To construct the column, we will present a cross section sequence. First we describe a \emph{down-shift} gadget.
That is, consider $i$, such that $H_{i+1} = H_i-2\varepsilon\cdot d < H_i$.
Figure~\ref{fig:level_shift_layers} shows the cross section evolution.
This cross section comprises of a two vertical lines separated by a top horizontal line.
The vertical lines are connected to the top segment with
a sequence of $2k$ horizontal segments that \emph{accordion} back and forth.
During each $1$-interval, all segments move along the positive $y$ direction (Figure~\ref{fig:level_shift0}),to create the $i$th level.
Subsequently, during the level shift, all segments move in the $x-z$ plane (Figure~\ref{fig:level_shift1},~\ref{fig:level_shift2}).
\begin{restatable}{pro}{AccordionEven}
\label{pro:accordion_even}
The number of accordion folds during horizontal evolution (along the $y$ axis) must be even.
\end{restatable}

The top segment moves downwards in intervals of $2\varepsilon$.
During this process, the horizontal segments move downwards continuously (Figure~\ref{fig:level_shift1},~\ref{fig:level_shift2}).
For the first $\varepsilon$ time interval, a new horizontal downwards moving accordion segment of length zero is created at both accordions,
and the vertical segments move towards each other along the $x$-axis (Figure~\ref{fig:level_shift1}).
For the next $\varepsilon$ time interval, similar (oppositely oriented) accordion segments are created at the lowest position.
This time, the vertical segments move outwards (Figure~\ref{fig:level_shift2}),
until they reach their original position (Figure~\ref{fig:level_shift3}).
Overall, two sets of accordion segments on either side are added, and the height of the top segment decreases by $2\varepsilon$.

In the case that $H_{i+1} = H_i+2\varepsilon\cdot d$, the level up-shift
is simply the down-shift evolution in reverse (Figure~\ref{fig:column_connector}).
This transition is only possible if the initial number of accordion segments in the $H_i$ cross section is at least $2d$.
Specifically, assuming that the minimum height is zero, we have the following lemma.
%This gives us the following lemma.
\begin{lemma}
\label{lem:layer_change}
If the number of accordion segments at level $H_i$ is $l_i$,
then the number of accordion segments after transitioning to level $H_{i+1}$ is $l_{i+1} = l_i - (H_{i+1}-H_i)/\varepsilon$.
Specifically, if the number of layers at level $0$ is $l$, then the number of layers
at level $\max\left\{ H_i\right\}$ is $l - \max\left\{ H_i\right\}/\varepsilon$.
%Specifically, if the number of layers at level $\min\left\{ H_i\right\}$ is $L$, then the number of layers
%at level $\max\left\{ H_i\right\}$ is $L - \left( \max\left\{ H_i\right\}-\min\left\{ H_i\right\} \right)/\varepsilon$.
\end{lemma}
\begin{corollary}
\label{cor:layer_limit}
Since the number of accordion segments can never be negative, the minimum number of layers at at level zero
is $L = \max\left\{ H_i\right\}/\varepsilon$. This also ensures that every other level shift is also possible.
%is $L = \left( \max\left\{ H_i\right\}-\min\left\{ H_i\right\} \right)/\varepsilon$. This also ensures that every other level shift is also possible.
\end{corollary}
\begin{corollary}
\label{cor:column_cross_section_length}
The length of the cross section at a zero level is at least $1 + 2\cdot L\cdot \varepsilon$.
So, the minimum possible length of the cross section under our construction is $1 + 2\cdot\max\left\{ H_i\right\}$.
\end{corollary}

This provides the minimum width of any strip required to construct a column extrusion.
Also the minimum required strip length is given by
$$ T = 1 + D_1  +  1 + D_2  +  1 + D_3  +\cdots\cdots +  1 + D_{m-1}  +  1 = m + \sum^{m-1}_{i=1} \left| H_{i+1}-H_i\right|. $$

\begin{restatable}{thm}{column_extrusion}
\label{thm:column_extrusion}
A given column extrusion with heights $\left\{ H_1, H_2,\cdots H_n \right\}$, can be constructed from a strip of paper with size
$X\times T$, where
\begin{align*}
X\ge 1 + 2\cdot\max\left\{ H_i\right\}, && T \ge \left( m + \sum\limits^{m-1}_{i=1} \left| H_{i+1}-H_i\right|\right).
\end{align*}
%X\ge 1 + 2\cdot\left( \max\left\{ H_i\right\}-\min\left\{ H_i\right\}\right) &&T \ge \left( m + \sum\limits^{m-1}_{i=1} \left| H_{i+1}-H_i\right|\right)
\end{restatable}
