\section{Cross Sections}
\label{sec:cross_sections}

We introduce a new method of origami construction, using cross section disgrams.
Instead of beginning our construction from a 2-dimensional sheet of paper,
we consider a 1-dimensional cross section moving forwards in time.
A simple example is demonstrated in Figure~\ref{fig:strip_narrowing}.
\begin{figure}[htpb]
    \centering
    \includegraphics[width=0.8\linewidth]{figures/strip_narrowing.jpg}
    \caption{A strip narrowing gadget constructed from cross sections.}
    \label{fig:strip_narrowing}
\end{figure}

\subsection{Evolution}
\label{sec:evolution}

\subsubsection*{Segments and Velocities}
\label{sec:segments_and_velocities}

\subsubsection*{Evolution of Adjacent Segments}
\label{sec:evolution_of_adjacent_segments}

\subsubsection*{Creation of New Segments}
\label{sec:creation_of_new_segments}

