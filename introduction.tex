\section{Introduction}
\label{sec:introduction}

Many algorithms and universality results exist for producing parameterized
families of origami structures, but few are provably efficient, i.e. provide
constructions from a paper having dimensions within a low constant factor of an
optimal construction. At 5OSME, \cite{MazeFolding_Origami5} presented a
efficient construction for folding orthogonal mazes which is computable in
polynomial time. Origamizer, presented in \cite{Origamizer_SoCG2017} constructs
foldings corresponding to general polyhedral surfaces, but does not provide any
bound on the efficiency of the constructions. On the other hand, TreeMaker from
~\cite{Lang} produces efficient crease patterns to fold uniaxial bases, but may
require exponential time to to find an efficient solution. 

In this paper, we present an algorithm for efficiently producing an origami
folding that corresponds to an input {\bf orthogonal terrain} with arbitrary
rational extrusion heights. A folding corresponds to an orthogonal terrain if
the folding covers every point on the terrain, but no point on the folding
exists above the terrain. This result improves an algorithm,
\cite{BoxPleating_Origami5} also presented at 5OSME, applicable to a more
general class of inputs, providing a universal construction to fold general
orthogonal polyhedra, though the construction is less inefficient than our
construction applied to orthogonal terrains. Our construction approach follows
three steps: 

\vspace{-0.2pc} 
\begin{enumerate} 
\item Decompose the orthogonal terrain into strips, constant along one dimension.
\item Cover the strips efficiently using rectangular strips of paper.
\item Stitch the strips together along matching boundaries.
\end{enumerate}
\vspace{-0.2pc}

In order to better communicate the algorithm and the final folded state
produced, we also introduce a new {\bf cross section evolution} representation
of a folded isometry: a straight line is swept across the crease pattern of a
folded surface, and we keep track of how the folding of the line evolves as a
cross section of the folded surface. The propagation of the cross section
between crease pattern vertices is uniquely determined by the initial
orientation of the cross section, so the folded isometry can be constructed by
sweeping the line and locally modifying the cross section when crossing crease
pattern vertices during propagation. This representation not only simplifies the
description of the 3D folded isometries constructed, but also provides a simpler
framework to argue that the folded state does not self intersect, by propagating
planar cross sections monotonically along a single direction. We then
show that our construction's efficiency is within a small constant factor of
any folding with optimal efficiency.
